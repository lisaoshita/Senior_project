\documentclass{beamer}
 
\usepackage[utf8]{inputenc}
\usepackage{lmodern}% http://ctan.org/pkg/lm
% \usepackage[demo]{graphicx}
 
 
%Information to be included in the title page:
\title{Predicting Airbnb User Booking Destinations}
\author{Lisa Oshita\\Dr. Glanz}


\begin{document}
 
\frame{\titlepage}

% background information about the competition
\begin{frame}
\frametitle{About the Competition}
  \begin{itemize}
    \item Recruiting competition hosted on Kaggle from November 2015 to February 2016
    \item Task: build a model to predict where new Airbnb users will book their first destinations
    \item 12 possible destinations to predict: Australia, Canada, France, Germany, Italy, Netherlands, Portugal, Spain, United Kingdom, US, other and no destination found (NDF)
  \end{itemize}
\end{frame}

% about train_users and sessions data
\begin{frame}
\frametitle{About the Data}
\begin{itemize}
  \item \texttt{train\_users}: 213,415 observations and 16 rows, contains information about users from 2010 to 2014
  \item \texttt{sessions}: 1,048,575 rows and 12,994 unique users, contains information about web session activity for each user
  \item 10\% of rows from each unique user were randomly sampled
  \item New sampled sessions data contained 104,424 rows
\end{itemize}
\end{frame}

% talk about the imbalanced classes - present a problem later in analysis 
\begin{frame}
\frametitle{Booking Destinations: extremely imbalanced classes}
\begin{table}[ht]
\centering
\begin{tabular}{| r |c |}
  \hline
  \textbf{Destination} & \textbf{Percentage of the data (\%)} \\ 
  \hline
  NDF & 58.35 \\ 
  US & 29.22 \\ 
  other & 4.73 \\ 
  FR & 2.35 \\ 
  IT & 1.33 \\ 
  GB & 1.09 \\ 
  ES & 1.05 \\ 
  CA & 0.67 \\ 
  DE & 0.50 \\ 
  NL & 0.36 \\ 
  AU & 0.25 \\ 
  PT & 0.10 \\ 
   \hline
\end{tabular}
\caption{Percentage of data each destination accounts for}
\label{table:countries}
\end{table}
\end{frame}

% models explored
% explain why I chose these three models 
\begin{frame}
\frametitle{Models}
  \begin{itemize}
    \item Extreme Gradient Boosting (XGBoost)
    \item Random forest
    \item Stacked model
  \end{itemize}
\end{frame}

% feature engineering 
\begin{frame}
\frametitle{Feature Engineering}
\begin{itemize}
  \item Date features
  \item Age and gender
  \item Count features created from the sessions data (314 features: number of times a user viewed recent reservations, number of times a user viewed similar listings...)
  \item Summary statistics of seconds elapsed for each user’s web session
  \item After all feature engineering and one-hot encoding, there were a total of 596 features for use in the model
\end{itemize}
\end{frame}

\begin{frame}
\frametitle{Model Building}
\begin{itemize}
  \item Full data was split into training and test sets
  \item 5-fold cross validation with both the XGBoost and Random forest achieved 87\% classification accuracy and NDCG score of 0.92, but only made predictions for NDF and the US
  \item Both models were fit to just the top 200 most important features and cross-validation was again performed - both achieved same results as previously, but computation time decreased 
\end{itemize}
\end{frame}


% \begin{frame}
% \frametitle{Model Building: Feature Importance}
% include tables of feature importance? 
% \end{frame}
% 
% \begin{frame}
% \frametitle{Model Building: Oversampling}
% \begin{itemize}
%   \item Oversampling with replacement from countries under-represented in the data, and undersample from countries over-represetned in the data
%   \item Synthetic Minority Oversampling Techniques (SMOTE)
% \end{itemize}
% \end{frame}
% 
% \begin{frame}
% \frametitle{Model Building: Results from Oversampling Techniques}
% \begin{itemize}
%   \item Same accuracy and NDCG scores as all previous models
%   \item Both random forest and XGBoost were now able to make predictions for all booking destinations (though not well) 
% \end{itemize}
% \end{frame}
% 
% % for this frame - make each bullet point pop up one at a time
% \begin{frame}
% \frametitle{Model Building: Stacked Model}
% \begin{itemize}
%   \item Partition the training data into five folds 
%   \item Build each model on the training folds, and test on the held-out test fold 
%   \item Store predictions from each model as two columns in the training data 
%   \item Repeat this process until each fold had been used as a test fold 
%   \item Fit each model to the full training data, and predict on the test set 
%   \item Store predictions as two columns in the test set 
%   \item Use a final XGBoost as the "stacker" - fit to the predictions stored in the training set, and predict on the predictions stored in the test set 
% \end{itemize}
% \end{frame}
% 
% \begin{frame}
% \frametitle{Model Building: Results of the Stacked Model}
% \begin{itemize}
%   \item Accuracy and NDCG scores were the same as all previous models 
%   \item include confusion matrix 
% \end{itemize}
% \end{frame}
% 
% \begin{frame}
% \frametitle{Model Building: Stacked Model confusion matrix}
% \begin{itemize}
%   \item include confusion matrix here 
% \end{itemize}
% \end{frame}
% 
% \begin{frame}
% \frametitle{Model Building: Final Models}
% \begin{itemize}
%   \item 5-fold cross-validation was performed for all three models on the entire data 
%   \item Accuracy and NDCG scores remained the same as all previous models 
%   \item Run times for the XGBoost, random forest and stacked models were 27.5 minutes, 28.7 minutes, and 3.24 hours, respectively (include this as a table?) 
% \end{itemize}
% \end{frame}
% 
% 
% \begin{frame}
% \frametitle{Discussion and Conclusions}
% \begin{itemize}
%   \item No effective strategy was found for improving model accuracy
%   \item The stacked model did not perform better than the two base models because both base models could not make accurate predictions for the under-represented booking destinations 
% \end{itemize}
% \end{frame}
% 
% \begin{frame}
% \frametitle{Next Steps?}
% \begin{itemize}
%   \item Stack more than just two models
%   \item Build additional models to predict and impute missing values 
%   \item Find a way to incorporate the other 4 remaining data sets 
%   \item Parameter tuning 
% \end{itemize}
% \end{frame}

\end{document}
