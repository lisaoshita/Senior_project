\documentclass{article}

\usepackage{textcomp} % for apostrophe

% for tables
\usepackage{multirow} 
\usepackage{array}
\newcolumntype{L}[1]{>{\raggedright\let\newline\\\arraybackslash\hspace{0pt}}m{#1}}
\usepackage{caption}

\title{Airbnb's New User Bookings Kaggle Competition}
\author{Lisa Oshita}

\usepackage{Sweave}
\begin{document}
\Sconcordance{concordance:final_report.tex:final_report.Rnw:%
1 7 1 1 0 93 1}


\maketitle

% -----------------------------------------------------------------------------------------------

\section{About the Competition}


Founded in 2008, Airbnb is an online accommodation marketplace featuring millions of houses,
rooms and apartments for rent or lease in over 200 countries. As part of a recruiting competition,
Airbnb partnered with Kaggle to host a data science competition starting in November 2015 and ending
in February 2016. The task of this competition was to build a model to accurately predict where new
Airbnb users will make their first bookings. Participants were given access to six data sets containing
information about the users and the 12 possible destinations to predict: Australia, Canada, France, 
Germany, Italy, Netherlands, Portugal, Spain, United Kingdom, US, other and no destination found (NDF). 
No destination found indicates that a user did not make a booking. Other indicates that a booking 
was made to a country not listed. This paper describes the process used to develop several models 
to predict booking destinations, with the goal of exploring and implementing machine learning concepts.


% -----------------------------------------------------------------------------------------------

\section{XGBoost, Random Forest, Stacked Models}

This paper explores three different models: Extreme gradient boosting (XGBoost), random forests, and stacked models. 

XGBoost algorithms are a fairly new method of supervised ensemble learning that performs consistently 
better than single-algorithm models. It is a form of gradient boosting that introduces a different, more 
formal method of regularization to prevent overfitting---enabling it to outperform other models. Additionally, 
XGBoost algorithms are parallelizable, allowing it to fully utilize the power of computers, which effectively 
decreases computation time. As XGBoosts have been used to place highly in Kaggle competitions and were 
also used by many of the top 100 participants of this Airbnb competition, this was the first model implemented.

Random forest models are another form of ensemble modeling frequently used in Kaggle competitions. 

Stacking, also called meta ensembling, is a technique used to combine information from individual predictive 
models to create a new model. Because stacked models are able to correct and build upon the performance of those
base models, they usually achieve better results. This technique is also frequently used in Kaggle competitions
and was used by many of the top participants of this competition. 

% -----------------------------------------------------------------------------------------------

\section{About the Data}

Of the six available data sets, two were utilized: train\_users and sessions. The sessions data, with 1,048,575 rows and 12,994 unique users, contains information regarding user's web session activity (e.g. what pages were viewed on the Airbnb site, what actions were taken by the user). To minimize computation time, 10\% of rows from each unique user was randomly sampled. This new sampled data, containing 104,424 rows, was used for all following analyses. The train\_users data, with 213,415 rows and 16 original columns, contained user information starting in 2010. Information includes features like age, gender, date of account creation, sign up method (Facebook, google...), and language preference. Exploratory analysis of this data revealed several factors. 

\begin{table}[ht]
\centering
\begin{tabular}{| l |l |}
  \hline
  \textbf{Destination} & \textbf{Percentage of Bookings (\%)} \\ 
  \hline
  NDF & 58.35 \\ 
  US & 29.22 \\ 
  other & 4.73 \\ 
  FR & 2.35 \\ 
  IT & 1.33 \\ 
  GB & 1.09 \\ 
  ES & 1.05 \\ 
  CA & 0.67 \\ 
  DE & 0.50 \\ 
  NL & 0.36 \\ 
  AU & 0.25 \\ 
  PT & 0.10 \\ 
   \hline
\end{tabular}
\caption{Percentage of Bookings Made to Each Destination}
\label{table:countries}
\end{table}

Table \ref{table:countries} shows how imbalanced the target destinations. Together, NDF and US
account for almost 90\% of the entire data, indicating that a majority of users either 
have yet to make a booking, or book to locations in the US. The remaining 10, under-represented 
destinations make up less than 5\% of the data each. The extremity of this imbalance will present 
challenges for the machine learning algorithms explored in this paper.

% -----------------------------------------------------------------------------------------------


\section{Feature Engineering}

From the train\_users data, 17 features were created. Date features were pulled apart into 
month, year, day of the week, and season features. The differences in days between date features 
(days between the date an account was created and the date of first booking, and days between the date a 
user was first active and the date of first booking) were also created. Missing and erroneous values in the gender feature 
were cleaned and recoded. The age feature contained several issues. 87,990 users (41\% of users) did 
not enter an age, while 2,345 users (1\% of users) entered an age of greater than 100. These values 
were recoded, while missing values were imputed using an XGBoost to predict those 
ages based on ages available in the data. 

From the sessions data, 320 features were created. The data was aggregated by each user and 
features counting the number of unique levels of each categorical feature were created. From secs\_elapsed, 
mean, median, standard deviation, minimum and maximums were calculated for each unique user. These features 
were then joined by user ID to the train\_users data. 

\begin{table}[!htbp]
\centering
\begin{tabular}{|L{2cm}|L{10cm}|}
  \hline
  \multirow{9}{2cm}{\textbf{Original Features}} & Sign up method (facebook, basic, google) \\ \cline{2-2}
  & Sign up flow (17 levels) \\ \cline{2-2}
  & Language preference (25 levels) \\ \cline{2-2}
  & Affiliate channel (8 levels: direct, remarketing, api...) \\ \cline{2-2}
  & Affiliate provider (18 levels: craigslist, bing, email-marketing...) \\ \cline{2-2}
  & First affiliate tracked (8 levels: untracked, linked, local ops...) \\ \cline{2-2}
  & Sign up app (Web, Moweb, iOS, Android) \\ \cline{2-2}
  & First device type (9 levels: Mac Desktop, iPad, iPhone...) \\ \cline{2-2}
  & First browser (52 levels: Chrome, Firefox, Safari...) \\ \hline
  \multirow{5}{2cm}{\textbf{Derived Features}} & Year, month, day of the week, season features of date account created, 
  date first active, date first booking\\ \cline{2-2}
  & Days between date account created and date of first booking \\ \cline{2-2}
  & Days between date first active and date of first booking \\ \cline{2-2}
  & Age \\ \cline{2-2}
  & Gender \\ \cline{2-2} 
  & Count features created from the sessions data (314 features: number of times a user viewed recent reservations, number of 
  times a user viewed similar listings...) \\ \cline{2-2}
  & Mean, median, standard deviation, minimum, maximum of seconds elapsed for each user's web activity \\ \hline
\end{tabular}
\caption{Original and Derived Features}
\label{table:features}
\end{table}

One-hot encoding was used to convert categorical features to a form compatible with machine learning algorithms.  
Essentially, a boolean column was generated for each level of a categorical feature. Continuous 
features were left as is. After one-hot encoding, there were a total of 596 features. Table 
\ref{table:features} displays all features, both original and derived, 
that were included in the training data for model building. 


\section{Model Building and Results}

Although the methods used for model building were circular and iterative processes, rather than linear. the following describes the general process used to develop the XGBoost, random forest, and stacked model. 


The full data was partitioned into one training set, containing 149,422 rows (70\% of the full data), and one 
test set containing 64,029 rows. All model building was performed on just the training set. For both the XGBoost 
and random forest models, five fold cross-validation was performed on the training set, including all 596 features. 
Both models achieved 87\% classification accuracy, as well as a Normalized Discounted Cumulative Gain (NDCG) score 
(the metric used in the actual competition) of 0.92. However, both models only made predictions for NDF, US, and 
other. Examining feature importance for each model showed that not all features were necessary. So, the models were 
refit to just the 200 most important features and cross-validation was performed again. While accuracy and NDCG
scores remained the same, computation time decreased tremendously. All following models were built only on just those 
200 features. 

\begin{table}[ht]
\centering
\begin{tabular}{| l | l | l |}
  \hline
  \textbf{Country} & \textbf{N} & \textbf{Proportion} \\ 
  \hline
  AU & 5670 & 0.05 \\ 
  CA & 5000 & 0.04 \\ 
  DE & 5944 & 0.05 \\ 
  ES & 6300 & 0.05 \\ 
  FR & 7034 & 0.06 \\ 
  GB & 6508 & 0.06 \\ 
  IT & 5955 & 0.05 \\ 
  NDF & 35000 & 0.30 \\ 
  NL & 5340 & 0.05 \\ 
  other & 7066 & 0.06 \\ 
  PT & 5320 & 0.05 \\ 
  US & 20000 & 0.17 \\ 
   \hline
\end{tabular}
\label{table:smote}
\end{table}

To account for the highly imbalanced classes, several over-sampling techniques were explored. Data for under-represented
destinations were over sampled with replacement and data from over-represented destinations were under sampled. 
Building and testing models on this new sampled training data resulted in worse predictive accuracy than previous
models. The final method settled on was Synthetic Minority Oversampling Techniques (SMOTE), in which underrepresented
classes are upsampled by with generated synthetic examples selected from the k-nearest neighbors of these under-represented 
destinations. This over-sampling technique was combined with undersampling of the over-represented destinations. The 
resulting training set contained 115,137 observations. Table \ref{table:smote} shows the number of observations and proportion
of data each destination accounts for, after SMOTE and undersampling was performed. 

The same model fitting processes (cross-validation with only the top 200 features) was again performed 
on this new training data. For both models, accuracy and NDCG scores remained the same. However, both models were 
able to make predictions for all countries (rather than just NDF, US and other), although predictions for these 
under-represented countries were not significantly accurate. Figures blah and blah show the confusion matrices of predictions
for the XGBoost and random forest fit to this new training data. Figures blah and blah display the feature importance 
for each model. 

% maybe include a visual explaining the flow of this stacked model 

Both models were then combined in a stacked model. The training data was partitioned into five folds, each 
fold making up around 20\% of the full training data. Each model was then built on the training folds, and 
tested on the held out fold. For example, the XGBoost was first built on folds one, two, three and four, and 
used to make predictions on fold five. In the second iteration, the XGBoost was build on folds two, three, 
four and five, and used to make predictions on fold one. This process was repeated for both models until each 
fold had been used as a test fold. Predictions from these models were stored as two columns in the training data. 
Then, each model was fit to the full training data (ignoring the folds and predictions created in the
previous step), and used to predict on the held out test set. Predictions from these models were stored in two 
columns in that test set. A final XGBoost was used as the model to combine, or stack the information from the 
previous processes. An XGBoost was fit to the predictions stored in the training set and tested on the predictions 
stored in the test set. Resulting accuracy and NDCG score for this stacked model was the same as its base models, 
as well as all previous models. 

\section{Discussion and Conclusion}

Although various strategies for improving model performance were explored, no effective strategy was found---each model achieved around 87\% accuracy and NDCG score of about 0.92. One explanation for this is how imbalanced the target classes are, as described above. Essentially, there is not enough information available about the under-represented destinations to enable models to make predictions for those underrepresented countries. Instead, each model classified them mainly as bookings 
to the US. Since NDF and US constitute such a large portion of the data, the accuracy of each model 
remained fairly high at 87\%. In order to create a model that can make accurate predictions for all 12 possible 
destinations, a strategy to resolve this issue of highly imbalanced classes would need to be implemented.

To explain why the stacked model did not result in better performance, the base models need to be examined. 
Stacked models generally perform well when its base models differ in performance. For example, suppose base model 1
could predict five countries well, but not the other 7. Suppose the opposite was true for base model 2: it could predict those 7 countries well, but not the other 5. In this case, a stacked model would be useful because of its ability to build upon and 
correct the base models' performances. But since both the random forest and XGBoost could only make accurate predictions 
for NDF and the US, the stacked model's performance was not a significant improvement. 

There are several possibilties for continued work on this project. One idea is to stack more than just two models, as the top three participants in this competition combined numerous algorithms in multiple layers. Another possible strategy, used by the second place participant of this competition, is to build additional models to predict and impute missing values for certain features. Missing values were a significant issue with several features, like age and gender. Building models to address this may be helpful with model performance. Another strategy is to find a way to incorporate the other two available data sets: countries, which contains summary statistics about the destinations, and age\_gender\_bkts, which contains statistics describing the users' age groups, gender, and destinations. The second place winner built models to predict features within the countries data, and use those features in the final destination predictions. Parameter tuning for each model, to optimize the way each model builds on the data, may also be helpful with model performance.  




\section{References}



























\end{document}
